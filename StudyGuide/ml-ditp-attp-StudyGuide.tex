\documentclass[12pt]{article}
%%%%%%%%%%%%%%%%%%%%%%%%%%%%%%%%%%%%%%%%%%%%%%%%%%%%%%%%%%%%%
% Lecture Specific Information to Fill Out
%%%%%%%%%%%%%%%%%%%%%%%%%%%%%%%%%%%%%%%%%%%%%%%%%%%%%%%%%%%%%
\newcommand{\LectureTitle}{Study Guide}
%\newcommand{\LectureDate}{\today}
\newcommand{\LectureDate}{\today}
\newcommand{\LectureClassName}{Advanced Topics in Theoretical Physics}
\newcommand{\LatexerName}{Juan Rojo}
%%%%%%%%%%%%%%%%%%%%%%%%%%%%%%%%%%%%%%%%%%%%%%%%%%%%%%%%%%%%%

% Change "article" to "report" to get rid of page number on title page
\usepackage{amsmath,amsfonts,amsthm,amssymb}
\usepackage{setspace}
\usepackage{Tabbing}
\usepackage{fancyhdr}
\usepackage{lastpage}
\usepackage{extramarks}
\usepackage{chngpage,url}
\usepackage{soul,color}
\usepackage{graphicx,float,wrapfig,amsmath}
\usepackage{afterpage,textcomp}
\usepackage{abstract}
\usepackage[version=3]{mhchem} % for chemical equations
\usepackage{chemfig} % for drawing chemical compounds
\usepackage{pgfplots} % for plotting equations

\usepackage{tcolorbox}

\usepackage{bbold}

\usepackage[colorlinks, citecolor=blue,anchorcolor=blue,menucolor=blue, 
linkcolor=blue,filecolor=blue,runcolor=blue,urlcolor=blue,frenchlinks=blue]{hyperref}

\newcommand{\la}{\left\langle}
\newcommand{\ra}{\right\rangle}
\newcommand{\lc}{\left[}
\newcommand{\rc}{\right]}
\newcommand{\lp}{\left(}
\newcommand{\rp}{\right)}

\newcommand{\bref}[1]{(\ref{#1})}

\newcommand{\eqn}[1]{(\ref{#1})}
\newcommand{\be}{\begin{equation}}
\newcommand{\ee}{\end{equation}}
\newcommand{\ben}{\begin{displaymath}}
\newcommand{\een}{\end{displaymath}}
\newcommand{\bea}{\begin{eqnarray}}
\newcommand{\eea}{\end{eqnarray}}
\newcommand{\bean}{\begin{eqnarray*}}
\newcommand{\eean}{\end{eqnarray*}}
\newcommand{\nn}{\nonumber \\}
\newcommand{\ba}{\begin{array}}
\newcommand{\ea}{\end{array}}
\newcommand{\bi}{\begin{itemize}}
\newcommand{\ei}{\end{itemize}}
\def\gsim{\mathrel{\rlap{\lower4pt\hbox{\hskip1pt$\sim$}}
    \raise1pt\hbox{$>$}}}         %greater than or approx. symbol
\def\lsim{\mathrel{\rlap{\lower4pt\hbox{\hskip1pt$\sim$}}
    \raise1pt\hbox{$<$}}}         %less than or approx. symbol

\input{header}
\usepackage{slashed}

% In case you need to adjust margins:
\topmargin=-0.45in
\evensidemargin=0in
\oddsidemargin=0in
\textwidth=6.5in
\textheight=9.0in
\headsep=0.25in

% Setup the header and footer
\pagestyle{fancy}
\lhead{\LatexerName}
\chead{\LectureClassName: \LectureTitle}
\rhead{\LectureDate}
\lfoot{\lastxmark}
\cfoot{}
\rfoot{Page\ \thepage\ of\ \pageref{LastPage}}
\renewcommand\headrulewidth{0.4pt}
\renewcommand\footrulewidth{0.4pt}

%%%%%%%%%%%%%%%%%%%%%%%%%%%%%%%%%%%%%%%%%%%%%%%%%%%%%%%%%%%%%
% Some tools
\newcommand{\enterTopicHeader}[1]{\nobreak\extramarks{#1}{#1 continued on next page\ldots}\nobreak
                                    \nobreak\extramarks{#1 (continued)}{#1 continued on next page\ldots}\nobreak}
\newcommand{\exitTopicHeader}[1]{\nobreak\extramarks{#1 (continued)}{#1 continued on next page\ldots}\nobreak
                                   \nobreak\extramarks{#1}{}\nobreak}

\renewcommand{\contentsname}{{\normalsize Table of Contents}}
\renewcommand{\abstractname}{\Large \LectureTitle}
\renewcommand{\absnamepos}{flushleft}

\numberwithin{equation}{section}

%%%%%%%%%%%%%%%%%%%%%%%%%%%%%%%%%%%%%%%%%%%%%%%%%%%%%%%%%%%%%

\begin{document}
\begin{spacing}{1.1}
$\quad$\\
  \vskip100pt
\hrule
\vskip20pt

  %%%%%%%%%%%%%%%%%%%%%%%%%%%%%%%%%%%%%%%%%%%%%%%%
\begin{figure}[h]
\begin{center}
  \includegraphics[scale=0.33]{logo.pdf}\\
  \includegraphics[scale=0.23,angle=-90]{nikhef.pdf}
\end{center}
\end{figure}
%%%%%%%%%%%%%%%%%%%%%%%%%%%%%%%%%%%%%%%%%%%%%%%%%


\begin{center}
  {\bf \Large  Machine Learning: \\[0.2cm]a New Toolbox for Theoretical Physics}\\[0.4cm]
   {\bf \large  D-ITP Advanced Topics in Theoretical Physics}\\[0.8cm]
  \end{center}

 
\begin{center}
  {\Large \bf Study Guide}\\
  current version: \today \\[0.3cm]

{\large  Course coordinator: Dr Juan Rojo (\href{mailto:j.rojo@vu.nl}{j.rojo@vu.nl}})
\end{center}



\vskip30pt
\hrule
\vskip20pt

\clearpage

\section{General overview}

Under the generic umbrella denomination of Machine Learning (or Artificial Intelligence) one finds a number of statistical algorithms with the common feature that they manage to solve a specific task by learning by example from large datasets, as opposed to following a set of predefined rules. In recent years, Machine Learning tools have demonstrated their impressive potential to tackle a range of challenging problems related to classification, optimization, discrimination, inter- and extrapolation, efficient parameterizations, and pattern generalization, among several others.

In this course, we present an overview of the applications of Machine Learning to various areas of theoretical physics, from particle and astroparticle physics phenomenology to condensed matter and string theory to name a few. The course will be a mixture of theoretical foundations, discussion of practical applications, and hands-on tutorials.

\section{Schedule}

There will be four lectures, each taking place between 11am and 1pm
at room H331 of Nikhef. In the corresponding afternoon the hands-on tutorial
will take place, always in room H331. So the schedule of the course is:

\begin{itemize}

\item Monday 18th of November 2019: Lecture (11am to 1pm) and Tutorial (2pm to 5pm).

\item Monday 25th of November 2019: Lecture (11am to 1pm) and Tutorial (2pm to 5pm).

\item Monday 2nd of December 2019: Lecture (11am to 1pm) and Tutorial (2pm to 5pm).

\item Monday 9th of December 2019: Lecture (11am to 1pm) and Tutorial (2pm to 5pm).

\end{itemize}  

The exam will take place on Monday 16th of December.
%
The requisites to obtain a pass in this module of the Advanced Topics in Theoretical Physics
course are described below.

\section{Examination}

The examination of the course will be composed by two parts:

\begin{itemize}

\item A short report (around 4 pages long) about a specific application
  of Machine Learning to a physics problem that you find interesting
  or relevant for your research.
  %
  This report can be writtem individually or in groups of at most three students.
  %
  You should describe the physics problem that one is trying to use,
  the details of the machine learning algorithm adopted, and how
  this improves over more traditional approaches.

  Enthusiastic students might want to include with their submission
a piece of code that implements and executes this ML algorithm for some
toy application that they find interesting, though this is not a formal
requirement to achieve a pass in the course.

This report should be submitted to the course coordinator by Sunday 15th of
December at the latest.

\item Monday 16th of December, start at 11am and continue until all of the presentations
have been completed

individual or groups

Presentation (10 min)

Physics content and the description of the ML algorithm used.
Connection with the topics covered in this course

\end{itemize}  

Note that both handing in the reports and presenting them
are requisites (necessary but not sufficient) to pass the course.


\section{Course contents}

Some of the contents that will be covered during the course are:

\begin{itemize}

\item Artificial neural networks

\item Supervised versus unsupervised learning

\item Under-learning and over-fitting  

\item Machine learning application for parametrization 

\item Machine learning applications for classification and discrimination

\item Machine learning applications for inter- and extrapolation
  
\item Machine learning applications for complex optimisation and search problems

\item Hyper-parameter optimisation

\item Training strategies: from genetic algorithms to gradient descent methods

\item Generative models and generative adversarial networks

\item Deep learning  

\item ....  
  
\end{itemize}
  
In addition, more topics can be added to the discussion if there is sufficient
interest from the participating students.


\section{Hands-on tutorials}

This course will contain four hands-on tutorial sessions following each
of the lectures.
%
The idea is for the participants to learn how to use, modify, and apply
some of the machine learning algorithms used in the course.
%
These will be implemented in {\tt Python} programs.
%
While the emphasis of the course is not on programming itself but rather
in the algorithms and their applications, fully benefiting from the tutorials
requires some knowledge of {\tt Python}.
%
Participants should therefore bring their own laptop to the tutorial sessions.

The code that will be used for the tutorials, together with all other course
materials such as the lecture slides, can be obtained from a GitHub
repository:
\begin{center}
  \url{https://github.com/juanrojochacon/ml-ditp-attp}
\end{center}
This repo will be updated during the course so student should remind to
pull the latest version before the lectures.

Many of the examples that we will use require specific {\tt Python} libraries
such as {\tt SciPy}, {\tt NumPy}, and {\tt  MatplotLib}.
%
Participants in the course should make sure that their local {\tt Python}
installation is up to date and that all relevant packages are installed.
%
This can be verified by executing some of the example programs that
will be found in the {\tt Code} folder.

There exist a large number of possible ways to install Python.
%
Perhaps a convenient one is via the {\tt Conda} package manager
\begin{center}
\url{https://docs.conda.io/projects/conda/en/latest/user-guide/install/}
\end{center}
which is available for Windows, macOS and Linus, but other package managers
such as {\tt Homebrew} or {\tt Pip} can also be used.
%
Regardless of the choice, an important recommendation is to use a single
package manager, since combining two or more usually leads to issues
with the {\tt Python} installation.

\section{Useful reading}

Here we provide a list of potentially useful references about the topic
of machine learning in physics in general, in some cases with emphasis
in theoretical and high energy physics.
%
The interested student will find in these references sufficient material if
she is interested in deepening some of the concepts that will be covered in the lectures.

Lectures on ML in high-energy physics => slides

\end{spacing}

\end{document}
